\documentclass[]{article}
\usepackage{lmodern}
\usepackage{amssymb,amsmath}
\usepackage{ifxetex,ifluatex}
\usepackage{fixltx2e} % provides \textsubscript
\ifnum 0\ifxetex 1\fi\ifluatex 1\fi=0 % if pdftex
  \usepackage[T1]{fontenc}
  \usepackage[utf8]{inputenc}
\else % if luatex or xelatex
  \ifxetex
    \usepackage{mathspec}
  \else
    \usepackage{fontspec}
  \fi
  \defaultfontfeatures{Ligatures=TeX,Scale=MatchLowercase}
\fi
% use upquote if available, for straight quotes in verbatim environments
\IfFileExists{upquote.sty}{\usepackage{upquote}}{}
% use microtype if available
\IfFileExists{microtype.sty}{%
\usepackage{microtype}
\UseMicrotypeSet[protrusion]{basicmath} % disable protrusion for tt fonts
}{}
\usepackage[margin=1in]{geometry}
\usepackage{hyperref}
\hypersetup{unicode=true,
            pdftitle={MLE Project},
            pdfauthor={Yiyi Xu},
            pdfborder={0 0 0},
            breaklinks=true}
\urlstyle{same}  % don't use monospace font for urls
\usepackage{color}
\usepackage{fancyvrb}
\newcommand{\VerbBar}{|}
\newcommand{\VERB}{\Verb[commandchars=\\\{\}]}
\DefineVerbatimEnvironment{Highlighting}{Verbatim}{commandchars=\\\{\}}
% Add ',fontsize=\small' for more characters per line
\usepackage{framed}
\definecolor{shadecolor}{RGB}{248,248,248}
\newenvironment{Shaded}{\begin{snugshade}}{\end{snugshade}}
\newcommand{\AlertTok}[1]{\textcolor[rgb]{0.94,0.16,0.16}{#1}}
\newcommand{\AnnotationTok}[1]{\textcolor[rgb]{0.56,0.35,0.01}{\textbf{\textit{#1}}}}
\newcommand{\AttributeTok}[1]{\textcolor[rgb]{0.77,0.63,0.00}{#1}}
\newcommand{\BaseNTok}[1]{\textcolor[rgb]{0.00,0.00,0.81}{#1}}
\newcommand{\BuiltInTok}[1]{#1}
\newcommand{\CharTok}[1]{\textcolor[rgb]{0.31,0.60,0.02}{#1}}
\newcommand{\CommentTok}[1]{\textcolor[rgb]{0.56,0.35,0.01}{\textit{#1}}}
\newcommand{\CommentVarTok}[1]{\textcolor[rgb]{0.56,0.35,0.01}{\textbf{\textit{#1}}}}
\newcommand{\ConstantTok}[1]{\textcolor[rgb]{0.00,0.00,0.00}{#1}}
\newcommand{\ControlFlowTok}[1]{\textcolor[rgb]{0.13,0.29,0.53}{\textbf{#1}}}
\newcommand{\DataTypeTok}[1]{\textcolor[rgb]{0.13,0.29,0.53}{#1}}
\newcommand{\DecValTok}[1]{\textcolor[rgb]{0.00,0.00,0.81}{#1}}
\newcommand{\DocumentationTok}[1]{\textcolor[rgb]{0.56,0.35,0.01}{\textbf{\textit{#1}}}}
\newcommand{\ErrorTok}[1]{\textcolor[rgb]{0.64,0.00,0.00}{\textbf{#1}}}
\newcommand{\ExtensionTok}[1]{#1}
\newcommand{\FloatTok}[1]{\textcolor[rgb]{0.00,0.00,0.81}{#1}}
\newcommand{\FunctionTok}[1]{\textcolor[rgb]{0.00,0.00,0.00}{#1}}
\newcommand{\ImportTok}[1]{#1}
\newcommand{\InformationTok}[1]{\textcolor[rgb]{0.56,0.35,0.01}{\textbf{\textit{#1}}}}
\newcommand{\KeywordTok}[1]{\textcolor[rgb]{0.13,0.29,0.53}{\textbf{#1}}}
\newcommand{\NormalTok}[1]{#1}
\newcommand{\OperatorTok}[1]{\textcolor[rgb]{0.81,0.36,0.00}{\textbf{#1}}}
\newcommand{\OtherTok}[1]{\textcolor[rgb]{0.56,0.35,0.01}{#1}}
\newcommand{\PreprocessorTok}[1]{\textcolor[rgb]{0.56,0.35,0.01}{\textit{#1}}}
\newcommand{\RegionMarkerTok}[1]{#1}
\newcommand{\SpecialCharTok}[1]{\textcolor[rgb]{0.00,0.00,0.00}{#1}}
\newcommand{\SpecialStringTok}[1]{\textcolor[rgb]{0.31,0.60,0.02}{#1}}
\newcommand{\StringTok}[1]{\textcolor[rgb]{0.31,0.60,0.02}{#1}}
\newcommand{\VariableTok}[1]{\textcolor[rgb]{0.00,0.00,0.00}{#1}}
\newcommand{\VerbatimStringTok}[1]{\textcolor[rgb]{0.31,0.60,0.02}{#1}}
\newcommand{\WarningTok}[1]{\textcolor[rgb]{0.56,0.35,0.01}{\textbf{\textit{#1}}}}
\usepackage{graphicx,grffile}
\makeatletter
\def\maxwidth{\ifdim\Gin@nat@width>\linewidth\linewidth\else\Gin@nat@width\fi}
\def\maxheight{\ifdim\Gin@nat@height>\textheight\textheight\else\Gin@nat@height\fi}
\makeatother
% Scale images if necessary, so that they will not overflow the page
% margins by default, and it is still possible to overwrite the defaults
% using explicit options in \includegraphics[width, height, ...]{}
\setkeys{Gin}{width=\maxwidth,height=\maxheight,keepaspectratio}
\IfFileExists{parskip.sty}{%
\usepackage{parskip}
}{% else
\setlength{\parindent}{0pt}
\setlength{\parskip}{6pt plus 2pt minus 1pt}
}
\setlength{\emergencystretch}{3em}  % prevent overfull lines
\providecommand{\tightlist}{%
  \setlength{\itemsep}{0pt}\setlength{\parskip}{0pt}}
\setcounter{secnumdepth}{0}
% Redefines (sub)paragraphs to behave more like sections
\ifx\paragraph\undefined\else
\let\oldparagraph\paragraph
\renewcommand{\paragraph}[1]{\oldparagraph{#1}\mbox{}}
\fi
\ifx\subparagraph\undefined\else
\let\oldsubparagraph\subparagraph
\renewcommand{\subparagraph}[1]{\oldsubparagraph{#1}\mbox{}}
\fi

%%% Use protect on footnotes to avoid problems with footnotes in titles
\let\rmarkdownfootnote\footnote%
\def\footnote{\protect\rmarkdownfootnote}

%%% Change title format to be more compact
\usepackage{titling}

% Create subtitle command for use in maketitle
\newcommand{\subtitle}[1]{
  \posttitle{
    \begin{center}\large#1\end{center}
    }
}

\setlength{\droptitle}{-2em}

  \title{MLE Project}
    \pretitle{\vspace{\droptitle}\centering\huge}
  \posttitle{\par}
    \author{Yiyi Xu}
    \preauthor{\centering\large\emph}
  \postauthor{\par}
      \predate{\centering\large\emph}
  \postdate{\par}
    \date{9/21/2018}


\begin{document}
\maketitle

\hypertarget{many-local-maxima}{%
\section{3.3.2 Many Local Maxima}\label{many-local-maxima}}

\hypertarget{find-the-log-likelihood-and-plot}{%
\subsection{Find the log-likelihood and
plot}\label{find-the-log-likelihood-and-plot}}

Equation:

\[L(\theta;x)=\prod_{i=1}^{n}f(x_{i};\theta)=(2\pi)^{-n}\prod_{i=1}^{n}[1-cos(x_i-\theta)]\]
\[l(\theta)=\ln L(\theta;X)=-n\ln(2\pi)+\sum_{i=1}^{n}\ln[1-cos(x_i-\theta)]\]

\begin{Shaded}
\begin{Highlighting}[]
\NormalTok{sample <-}\StringTok{ }\KeywordTok{c}\NormalTok{(}\FloatTok{3.91}\NormalTok{, }\FloatTok{4.85}\NormalTok{, }\FloatTok{2.28}\NormalTok{, }\FloatTok{4.06}\NormalTok{, }\FloatTok{3.70}\NormalTok{, }\FloatTok{4.04}\NormalTok{, }\FloatTok{5.46}\NormalTok{, }\FloatTok{3.53}\NormalTok{, }\FloatTok{2.28}\NormalTok{, }\FloatTok{1.96}\NormalTok{,}
       \FloatTok{2.53}\NormalTok{, }\FloatTok{3.88}\NormalTok{, }\FloatTok{2.22}\NormalTok{, }\FloatTok{3.47}\NormalTok{, }\FloatTok{4.82}\NormalTok{, }\FloatTok{2.46}\NormalTok{, }\FloatTok{2.99}\NormalTok{, }\FloatTok{2.54}\NormalTok{, }\FloatTok{0.52}\NormalTok{)}
\NormalTok{l <-}\StringTok{ }\ControlFlowTok{function}\NormalTok{(x, y)\{}
  \DecValTok{-19}\OperatorTok{*}\KeywordTok{log}\NormalTok{(}\DecValTok{2}\OperatorTok{*}\NormalTok{pi)}\OperatorTok{+}\KeywordTok{sum}\NormalTok{(}\KeywordTok{log}\NormalTok{(}\DecValTok{1}\OperatorTok{-}\NormalTok{(}\KeywordTok{cos}\NormalTok{(x}\OperatorTok{-}\NormalTok{y))))}
\NormalTok{\}}
\NormalTok{theta <-}\StringTok{ }\KeywordTok{seq}\NormalTok{(}\OperatorTok{-}\NormalTok{pi, pi, }\FloatTok{0.1}\NormalTok{)}
\NormalTok{L <-}\StringTok{ }\KeywordTok{numeric}\NormalTok{(}\DecValTok{0}\NormalTok{)}
\ControlFlowTok{for}\NormalTok{ (i }\ControlFlowTok{in} \DecValTok{1}\OperatorTok{:}\StringTok{ }\KeywordTok{length}\NormalTok{(theta))\{}
\NormalTok{   L[i] <-}\StringTok{ }\KeywordTok{l}\NormalTok{(sample,theta[i])}
\NormalTok{\}}
\NormalTok{ sp=}\KeywordTok{spline}\NormalTok{(theta,L,}\DataTypeTok{n=}\DecValTok{1000}\NormalTok{)}
 \KeywordTok{plot}\NormalTok{(sp,}\DataTypeTok{col=}\StringTok{"red"}\NormalTok{,}\DataTypeTok{type=}\StringTok{"l"}\NormalTok{,}\DataTypeTok{xlim=}\KeywordTok{c}\NormalTok{(}\OperatorTok{-}\NormalTok{pi,pi),}\DataTypeTok{ylim=}\KeywordTok{c}\NormalTok{(}\OperatorTok{-}\DecValTok{80}\NormalTok{,}\OperatorTok{-}\DecValTok{25}\NormalTok{),}\DataTypeTok{lwd=}\DecValTok{2}\NormalTok{,}\DataTypeTok{xlab=}\StringTok{"theta"}\NormalTok{,}\DataTypeTok{ylab=}\StringTok{"l(theta)"}\NormalTok{,}\DataTypeTok{sub=}\StringTok{"loglikehood function against theta"}\NormalTok{,}\DataTypeTok{col.main=}\StringTok{"green"}\NormalTok{,}\DataTypeTok{font.main=}\DecValTok{2}\NormalTok{)}
\end{Highlighting}
\end{Shaded}

\includegraphics{MLE_Project_files/figure-latex/unnamed-chunk-1-1.pdf}

\hypertarget{find-method-of-moments-estimator-of-theta}{%
\subsection{Find Method of moments estimator of
theta}\label{find-method-of-moments-estimator-of-theta}}

\[E[X|\theta]=...=\sin(\theta)+\pi\]
\[\tilde{\theta}_n = \arcsin(\bar{X}_n - \pi)\]

\hypertarget{find-the-mle-by-newton-raphson-method}{%
\subsection{Find the MLE by Newton-Raphson
method}\label{find-the-mle-by-newton-raphson-method}}

\[l'(\theta)=\sum_{i=1}^n \frac{-sin(x_i-\theta)}{1-cos(x_i-\theta)}  \]
\[l''(\theta)=\sum_{i=1}^n \frac{1}{cos(x_i-\theta)-1}\]

\begin{Shaded}
\begin{Highlighting}[]
\NormalTok{l1 <-}\StringTok{ }\ControlFlowTok{function}\NormalTok{(sample, theta)\{}
  \KeywordTok{sum}\NormalTok{((}\OperatorTok{-}\KeywordTok{sin}\NormalTok{(sample}\OperatorTok{-}\NormalTok{theta))}\OperatorTok{/}\NormalTok{(}\DecValTok{1} \OperatorTok{-}\KeywordTok{cos}\NormalTok{(sample}\OperatorTok{-}\NormalTok{theta)))   }
\NormalTok{\}}
\NormalTok{l2 <-}\StringTok{ }\ControlFlowTok{function}\NormalTok{(sample,theta)\{}
  \KeywordTok{sum}\NormalTok{((}\DecValTok{1}\NormalTok{)}\OperatorTok{/}\NormalTok{(}\KeywordTok{cos}\NormalTok{(sample}\OperatorTok{-}\NormalTok{theta)}\OperatorTok{-}\DecValTok{1}\NormalTok{))   }
\NormalTok{\}}
\NormalTok{Newton.Method <-}\StringTok{ }\ControlFlowTok{function}\NormalTok{(y,f,f1)\{}
\NormalTok{   y0 <-}\StringTok{ }\NormalTok{y}
   \ControlFlowTok{for}\NormalTok{(i }\ControlFlowTok{in} \DecValTok{1}\OperatorTok{:}\DecValTok{100}\NormalTok{)\{}
\NormalTok{    y1 <-}\StringTok{ }\NormalTok{y0 }\OperatorTok{-}\StringTok{ }\KeywordTok{f}\NormalTok{(sample,y0)}\OperatorTok{/}\KeywordTok{f1}\NormalTok{(sample,y0)}
    \ControlFlowTok{if}\NormalTok{(}\KeywordTok{abs}\NormalTok{(y1}\OperatorTok{-}\NormalTok{y0)}\OperatorTok{<}\FloatTok{0.0001}\NormalTok{) }
      \ControlFlowTok{break}
\NormalTok{    y0 <-}\StringTok{ }\NormalTok{y1}
\NormalTok{   \}}
   \KeywordTok{return}\NormalTok{(}\KeywordTok{data.frame}\NormalTok{(}\DataTypeTok{init=}\NormalTok{y,}\DataTypeTok{root=}\NormalTok{y0,}\DataTypeTok{iter=}\NormalTok{i))}
\NormalTok{\}}

\KeywordTok{Newton.Method}\NormalTok{(pi }\OperatorTok{-}\StringTok{ }\KeywordTok{asin}\NormalTok{(}\KeywordTok{mean}\NormalTok{(sample) }\OperatorTok{-}\StringTok{ }\NormalTok{pi),l1,l2)}
\end{Highlighting}
\end{Shaded}

\begin{verbatim}
##       init     root iter
## 1 3.046199 3.170713    5
\end{verbatim}

\begin{Shaded}
\begin{Highlighting}[]
\KeywordTok{Newton.Method}\NormalTok{(}\KeywordTok{asin}\NormalTok{(}\KeywordTok{mean}\NormalTok{(sample) }\OperatorTok{-}\StringTok{ }\NormalTok{pi),l1,l2)}
\end{Highlighting}
\end{Shaded}

\begin{verbatim}
##         init        root iter
## 1 0.09539407 0.003136419    3
\end{verbatim}

\hypertarget{solutions}{%
\subsection{solutions}\label{solutions}}

\hypertarget{repoeat}{%
\subsection{repoeat}\label{repoeat}}

\hypertarget{modeling-beetle-data}{%
\section{3.3.3 Modeling beetle data}\label{modeling-beetle-data}}

\begin{Shaded}
\begin{Highlighting}[]
\NormalTok{beetles <-}\StringTok{ }\KeywordTok{data.frame}\NormalTok{(}
    \DataTypeTok{days    =} \KeywordTok{c}\NormalTok{(}\DecValTok{0}\NormalTok{,  }\DecValTok{8}\NormalTok{,  }\DecValTok{28}\NormalTok{,  }\DecValTok{41}\NormalTok{,  }\DecValTok{63}\NormalTok{,  }\DecValTok{69}\NormalTok{,   }\DecValTok{97}\NormalTok{, }\DecValTok{117}\NormalTok{,  }\DecValTok{135}\NormalTok{,  }\DecValTok{154}\NormalTok{),}
    \DataTypeTok{beetles =} \KeywordTok{c}\NormalTok{(}\DecValTok{2}\NormalTok{, }\DecValTok{47}\NormalTok{, }\DecValTok{192}\NormalTok{, }\DecValTok{256}\NormalTok{, }\DecValTok{768}\NormalTok{, }\DecValTok{896}\NormalTok{, }\DecValTok{1120}\NormalTok{, }\DecValTok{896}\NormalTok{, }\DecValTok{1184}\NormalTok{, }\DecValTok{1024}\NormalTok{))}
\end{Highlighting}
\end{Shaded}


\end{document}
